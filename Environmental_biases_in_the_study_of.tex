% pandoc-xnos: cleveref fakery
\newcommand{\plusnamesingular}{}
\newcommand{\starnamesingular}{}
\newcommand{\xrefname}[1]{\protect\renewcommand{\plusnamesingular}{#1}}
\newcommand{\Xrefname}[1]{\protect\renewcommand{\starnamesingular}{#1}}
\providecommand{\cref}{\plusnamesingular~\ref}
\providecommand{\Cref}{\starnamesingular~\ref}
\providecommand{\crefformat}[2]{}
\providecommand{\Crefformat}[2]{}

% pandoc-xnos: cleveref formatting
\crefformat{figure}{fig.~#2#1#3}
\Crefformat{figure}{Figure~#2#1#3}

\hypertarget{introduction}{%
\section{Introduction}\label{introduction}}

Ecological networks are a useful way to think about ecological systems
in which species or organism interact (Poisot et al. 2016c; Delmas et
al. 2018), and recently there has been an explosion of interest in their
dynamics across large temporal scales (Tylianakis \& Morris 2017; Baiser
et al. 2019), especially alongside environmental gradients (Trøjelsgaard
\& Olesen 2016; Pellissier et al. 2017). As ecosystems and climates are
changing rapidly, ecologists realized that networks are at risk or
unravelling, being invaded by exotic species that can destabilize them
(Strong \& Leroux 2014; Magrach et al. 2017), or adopt entirely novel
configurations (Guiden et al. 2019; Hui \& Richardson 2019). Simulation
studies seem to suggest that knowing the shape of the extant network is
not sufficient (Thompson \& Gonzalez 2017), and that it needs to be
supplemented by additional data on species properties, climate, and
climate projection.

This renewed interest in ecological networks has prompted several
methodological efforts. First, an expansion of the analytical tools to
study ecological networks and their variation in space (Poisot et al.
2012, 2015, 2017). Second, improvements in large-scale data-collection,
through increased adoption of molecular biology tools (Evans et al.
2016; Eitzinger et al. 2019) and crowd-sourcing of data collection
(Pocock et al. 2015; Bahlai \& Landis 2016; Roy et al. 2016). Finally, a
surge in the development of tools that allow to \emph{infer} species
interaction (Morales-Castilla et al. 2015) based on limited but
complementary data on existing network properties (Stock et al. 2017),
species traits (Gravel et al. 2013; Desjardins-Proulx et al. 2017;
~Brousseau et al. 2017; Bartomeus et al. 2016), and environmental
conditions (Gravel et al. 2018). These approaches tend to perform well
in data-poor environments (Beauchesne et al. 2016), and can be combined
through ensemble modeling or model averaging to generate possibly more
robust predictions (Pomeranz et al. 2018).

The common goal to these efforts is to facilitate the prediction of
network structure, both extant {[}Poisot et al. (2016b); \textbf{MARINE
FOODWEB}{]} and future (Albouy et al. 2014), and to appraise its
possible variation in response to changes. All of these developments
also share the need to be supported by state of the art data management:
novel quantitative tools demand a higher volume of network data; novel
collection techniques demand powerful data repositories; novel inference
tools demand easier integration between different types of data,
including but not limited to interactions, species traits, taxonomy,
occurrences, and local bioclimatic conditions.

Poisot et al. (2016a) -- overview of original DB + updates,
\textbf{TODO} map of networks + networks over time

\begin{itemize}
\tightlist
\item
  New data
\item
  number and amount of new information
\item
  web API for better data access, and two packages (one in Julia, the
  other in R) for accessing these data.
\item
  Mangal in its current form offers open network data that is ready to
  support synthesis at many scales.
\end{itemize}

Borrett et al. (2014) identified network ecology as one of the fastest
growing sub-field in the ecological sciences.

Synthesizing ecological data presents important challenges and also some
exciting opportunities. Mangal is well suited to offer such
opportunities in the study of ecological networks.

\begin{itemize}
\tightlist
\item
  A major challenge to ecological synthesis is generalizing from samples
  to the behaviour of ecological systems
\item
  two obstacles to such generalizing in ecological systems: data
  coverage and data quality

  \begin{itemize}
  \tightlist
  \item
    data coverage: are data collected from every relevant system?
  \item
    data quality: are data fit-for-purpose? Two particular aspects of
    quality

    \begin{itemize}
    \tightlist
    \item
      taxonomic resolution
    \item
      sampling effort
    \end{itemize}
  \end{itemize}
\end{itemize}

\textbf{Main question}, is the data fit for purpose, what can we do and
cannot do with it?

\hypertarget{global-trends-in-ecological-networks-description}{%
\section{Global trends in ecological networks
description}\label{global-trends-in-ecological-networks-description}}

\hypertarget{network-coverage-is-accelerating-but-spatially-biased}{%
\subsection{Network coverage is accelerating but spatially
biased}\label{network-coverage-is-accelerating-but-spatially-biased}}

\begin{figure}
\centering
\includegraphics{figures/figure_01_a.png}
\caption{Cumulative number of ecological networks available in
\texttt{mangal.io} as a function of the date of collection. About 1000
unique networks have been collected between 1987 and 2017, a rate of
just over 30 networks a year.\label{fig:temporal}}
\end{figure}

The earliest recorded ecological networks date back to the late
nineteenth century, with a strong increase in the rate of collection
around the 1980s (\xrefname{fig.}\cref{fig:temporal}). Although the
volume of available networks has increased over time, the sampling of
these networks in space has been uneven. In
\xrefname{fig.}\cref{fig:spatial}, we show that globally, network
collection is biased towards the Northern hemisphere, and than different
types of interactions have been sampled in different places. As such, it
is very difficult to find a spatial area of sufficiently large size in
which we have networks of predation, parasitism, and mutualism. The
inter-tropical zone is particularly data-poor, either because data
producers from the global South correctly perceive massive re-use of
their data by Western world scientists as a form of scientific
neo-colonialism (as advanced by Mauthner \& Parry 2013), thereby
providing a powerful incentive \emph{against} their publication, or
because ecological networks are subject to the same data deficit that is
affecting all fields on ecology in the tropics (Collen et al. 2008). As
Bruna (2010) identified almost ten years ago, improved data deposition
requires an infrastructure to ensure they can be repurposed for future
research, which we argue is provided by \texttt{mangal.io} for
ecological interactions.

\begin{figure}
\centering
\includegraphics{figures/figure_01_c.png}
\caption{Each point on the map corresponds to a network with parasitic,
mutualistic, and predatory interactions. It is noteworty that the
spatial coverage of these types of interactions is uneven; the Americas
have almost no parasitic network, for example. Some places have barely
been studied at all, including Africa and Eastern Asia. This
concentration of networks around rich countries speaks to an inadequate
coverage of the diversity of landscapes on Earth.\label{fig:spatial}}
\end{figure}

\hypertarget{different-interaction-types-have-been-studied-in-different-biomes}{%
\subsection{Different interaction types have been studied in different
biomes}\label{different-interaction-types-have-been-studied-in-different-biomes}}

Whittaker (1962) suggested that natural communities can be partitioned
across biomes, largely defined as a function of their relative
precipitation and temperature; in \xrefname{fig.}\cref{fig:biomes}, we
show that even though networks, overall, capture the diversity of the
precipitation/temperature climate well, types of networks have been
studied in sub-spaces only. Specifically, parasitism networks have been
studied in colder and drier climates; mutualism networks in wetter
climates; predation networks display less of a bias.

\begin{figure}
\centering
\includegraphics{figures/figure_02.png}
\caption{List of networks across biomes\label{fig:biomes}}
\end{figure}

\hypertarget{some-locations-on-earth-have-no-climate-analogue}{%
\subsection{Some locations on Earth have no climate
analogue}\label{some-locations-on-earth-have-no-climate-analogue}}

Climate analogue

\begin{figure}
\centering
\includegraphics{figures/figure_03_b.png}
\caption{tk\label{fig:analog}}
\end{figure}

\hypertarget{mutualistic-networks-are-biased-towards-more-unique-environments}{%
\subsection{Mutualistic networks are biased towards more unique
environments}\label{mutualistic-networks-are-biased-towards-more-unique-environments}}

\begin{figure}
\centering
\includegraphics{figures/figure_05_b.png}
\caption{tk\label{fig:ecc}}
\end{figure}

\hypertarget{conclusions}{%
\section{Conclusions}\label{conclusions}}

This means that we have surprisingly little information on ecological
networks dating before the acceleration of the climate crisis, and
therefore lack a robust baseline. Dalsgaard et al. (2013) provide strong
evidence that the extant shape of ecological networks emerged in part in
response to historical trends in climate change; in this perspective,
the lack of reference data before the acceleration of the effects of
climate change is of particular concern, as we may be deriving
intuitions on ecological networks structure and assembly rule from
networks that are in the midst of important ecological disturbances.
Although there are some research on the response of co-occurrence and
indirect interactions to climate change (Araújo et al. 2011; Losapio \&
Schöb 2017), these are a far cry from actual direct interactions;
similarly, the data on ``paleo-foodwebs'', \emph{i.e.} from deep
evolutionary time (Nenzén et al. 2014; Yeakel et al. 2014; Muscente et
al. 2018) represent the effect of more progressive change, and may not
adequately inform us about the future of ecological networks under sever
climate change.

\hypertarget{reducing-uncertainty-through-analogues}{%
\subsection{reducing uncertainty through
`analogues'}\label{reducing-uncertainty-through-analogues}}

When we lack direct observation of a community, often we must resort to
the use of `analog' communities -- that is, communities which are
similar in space or environment which have been sampled.

\begin{itemize}
\tightlist
\item
  Communities may be similar in at least two ways -- close in space, or
  close in climate
\item
  similarity may result in some (?) similarity in network structure,
  even if species different.
\item
  Always some uncertainty in such comparisons
\item
  reflects the need for more data gathering, can be used to target
  efforts
\end{itemize}

\hypertarget{future-of-network-ecology}{%
\subsection{Future of network ecology}\label{future-of-network-ecology}}

Use this spatial gaps for sampling recommendations

\hypertarget{more-complete-analyses}{%
\subsection{more complete analyses}\label{more-complete-analyses}}

We have only shown some high-level summaries of the data here; many
possibilities remain.

\hypertarget{more-data-collection}{%
\subsection{more data collection}\label{more-data-collection}}

We have demonstrated the considerable coverage of Mangal; however, our
summary also highlights important data-collection needs. In particular,
we need better information about (mutualists, desert food webs?)

\hypertarget{active-development-and-data-contribution}{%
\subsection{Active development and data
contribution}\label{active-development-and-data-contribution}}

This is an open-source project: all data and all code supporting this
are available on the Mangal project GitHub organization. Our hope is
that the success of this project will encourage similar efforts within
other parts of the ecological community. In addition, we hope that this
project will encourage the recognition of the contribution that software
creators make to ecological research.

\hypertarget{data-quality-sampling-effort-and-taxonomy}{%
\subsection{Data quality: sampling effort and
taxonomy}\label{data-quality-sampling-effort-and-taxonomy}}

Sampling effort and taxonomic detail are two very challenging but
important part of any ecological dataset. The datasets in Mangal
represent some of the most detailed studies of ecological networks
available. * measures of network structure may be particularly sensitive
to the amount of sampling effort * repeat sampling may be necessary to
capture a ``saturation'' of interactions. * we present some
visualization of the sampling coverage of Mangal {[}tk{]} * High
taxonomic resolution is difficult to achieve in ecology, especially
depending on the sampling method used (e.g.~gut contents vs
observations). We present a breakdown of the taxonomic resolution of
Mangal. * Ecological networks occur in various kinds, but they are not
all equally well sampled. We present a breakdown of the number of
parasitic, mutualistic and predator-prey networks sampled in Mangal

\hypertarget{references}{%
\section*{References}\label{references}}
\addcontentsline{toc}{section}{References}

\hypertarget{refs}{}
\leavevmode\hypertarget{ref-AlboVele14}{}%
\textbf{Albouy et al.} (2014). From projected species distribution to
food-web structure under climate change. \emph{Global Change Biology.}
20:730--41.

\leavevmode\hypertarget{ref-ArauRoze11}{}%
\textbf{Araújo et al.} (2011). Using species co-occurrence networks to
assess the impacts of climate change. \emph{Ecography.} 34:897--908.

\leavevmode\hypertarget{ref-BahlLand16}{}%
\textbf{Bahlai \& Landis}. (2016). Predicting plant attractiveness to
pollinators with passive crowdsourcing. \emph{Royal Society Open
Science.} 3:150677.

\leavevmode\hypertarget{ref-BaisGrav19}{}%
\textbf{Baiser et al.} (2019). Ecogeographical rules and the
macroecology of food webs. \emph{Global Ecology and Biogeography.} 0.

\leavevmode\hypertarget{ref-BartGrav16}{}%
\textbf{Bartomeus et al.} (2016). A common framework for identifying
linkage rules across different types of interactions. \emph{Funct Ecol.}
30:1894--903.

\leavevmode\hypertarget{ref-BeauDesj16}{}%
\textbf{Beauchesne et al.} (2016). Thinking Outside the Box--predicting
Biotic Interactions in Data-poor Environments. \emph{Vie et milieu-life
and enVironment.} 66:333--42.

\leavevmode\hypertarget{ref-BorrMood14}{}%
\textbf{Borrett et al.} (2014). The rise of Network Ecology: Maps of the
topic diversity and scientific collaboration. \emph{Ecological
Modelling.} 293:111--27.

\leavevmode\hypertarget{ref-BrouGrav17}{}%
\textbf{Brousseau et al.} (2017). Trait-matching and phylogeny as
predictors of predator-prey interactions involving ground beetles.
\emph{Functional Ecology.}

\leavevmode\hypertarget{ref-Brun10}{}%
\textbf{Bruna}. (2010). Scientific Journals can Advance Tropical Biology
and Conservation by Requiring Data Archiving. \emph{Biotropica.}
42:399--401.

\leavevmode\hypertarget{ref-CollRam08}{}%
\textbf{Collen et al.} (2008). The Tropical Biodiversity Data Gap:
Addressing Disparity in Global Monitoring. \emph{Tropical Conservation
Science.} 1:75--88.

\leavevmode\hypertarget{ref-DalsTroj13}{}%
\textbf{Dalsgaard et al.} (2013). Historical climate-change influences
modularity and nestedness of pollination networks. \emph{Ecography.}
36:1331--40.

\leavevmode\hypertarget{ref-DelmBess18}{}%
\textbf{Delmas et al.} (2018). Analysing ecological networks of species
interactions. \emph{Biological Reviews.}:112540.

\leavevmode\hypertarget{ref-DesjLaig17}{}%
\textbf{Desjardins-Proulx et al.} (2017). Ecological interactions and
the Netflix problem. \emph{PeerJ.} 5.

\leavevmode\hypertarget{ref-EitzAbre19}{}%
\textbf{Eitzinger et al.} (2019). Assessing changes in arthropod
predator--prey interactions through DNA-based gut content
analysis---variable environment, stable diet. \emph{Molecular Ecology.}
28:266--80.

\leavevmode\hypertarget{ref-EvanKits16}{}%
\textbf{Evans et al.} (2016). Merging DNA metabarcoding and ecological
network analysis to understand and build resilient terrestrial
ecosystems. \emph{Functional Ecology.}

\leavevmode\hypertarget{ref-GravBais18}{}%
\textbf{Gravel et al.} (2018). Bringing Elton and Grinnell together: a
quantitative framework to represent the biogeography of ecological
interaction networks. \emph{Ecography.} 0.

\leavevmode\hypertarget{ref-GravPois13}{}%
\textbf{Gravel et al.} (2013). Inferring food web structure from
predator-prey body size relationships. Freckleton, ed. \emph{Methods in
Ecology and Evolution.} 4:1083--90.

\leavevmode\hypertarget{ref-GuidBart19}{}%
\textbf{Guiden et al.} (2019). Predator--Prey Interactions in the
Anthropocene: Reconciling Multiple Aspects of Novelty. \emph{Trends in
Ecology \& Evolution.} 0.

\leavevmode\hypertarget{ref-HuiRich19}{}%
\textbf{Hui \& Richardson}. (2019). How to Invade an Ecological Network.
\emph{Trends in Ecology \& Evolution.} 34:121--31.

\leavevmode\hypertarget{ref-LosaScho17}{}%
\textbf{Losapio \& Schöb}. (2017). Resistance of plant--plant networks
to biodiversity loss and secondary extinctions following simulated
environmental changes. \emph{Functional Ecology.} 31:1145--52.

\leavevmode\hypertarget{ref-MagrHolz17}{}%
\textbf{Magrach et al.} (2017). Plant-pollinator networks in
semi-natural grasslands are resistant to the loss of pollinators during
blooming of mass-flowering crops. \emph{Ecography.}:n/a--a.

\leavevmode\hypertarget{ref-MautParr13}{}%
\textbf{Mauthner \& Parry}. (2013). Open Access Digital Data Sharing:
Principles, Policies and Practices. \emph{Social Epistemology.}
27:47--67.

\leavevmode\hypertarget{ref-MoraMati15}{}%
\textbf{Morales-Castilla et al.} (2015). Inferring biotic interactions
from proxies. \emph{Trends in Ecology \& Evolution.}

\leavevmode\hypertarget{ref-MuscPrab18}{}%
\textbf{Muscente et al.} (2018). Quantifying ecological impacts of mass
extinctions with network analysis of fossil communities.
\emph{PNAS.}:201719976.

\leavevmode\hypertarget{ref-NenzMont14}{}%
\textbf{Nenzén et al.} (2014). The Impact of 850,000 Years of Climate
Changes on the Structure and Dynamics of Mammal Food Webs. \emph{PLOS
ONE.} 9:e106651.

\leavevmode\hypertarget{ref-PellAlbo17}{}%
\textbf{Pellissier et al.} (2017). Comparing species interaction
networks along environmental gradients. \emph{Biol Rev Camb Philos Soc.}

\leavevmode\hypertarget{ref-PocoRoy15}{}%
\textbf{Pocock et al.} (2015). The Biological Records Centre: a pioneer
of citizen science. \emph{Biol J Linn Soc.} 115:475--93.

\leavevmode\hypertarget{ref-PoisBais16}{}%
\textbf{Poisot et al.} (2016a). mangal - making ecological network
analysis simple. \emph{Ecography.} 39:384--90.

\leavevmode\hypertarget{ref-PoisCana12}{}%
\textbf{Poisot et al.} (2012). The dissimilarity of species interaction
networks. \emph{Ecol Lett.} 15:1353--61.

\leavevmode\hypertarget{ref-PoisGrav16}{}%
\textbf{Poisot et al.} (2016b). Synthetic datasets and community tools
for the rapid testing of ecological hypotheses. \emph{Ecography.}
39:402--8.

\leavevmode\hypertarget{ref-PoisGuev17}{}%
\textbf{Poisot et al.} (2017). Hosts, parasites and their interactions
respond to different climatic variables. \emph{Global Ecol
Biogeogr.}:n/a--a.

\leavevmode\hypertarget{ref-PoisStou15}{}%
\textbf{Poisot et al.} (2015). Beyond species: why ecological
interaction networks vary through space and time. \emph{Oikos.}
124:243--51.

\leavevmode\hypertarget{ref-PoisStou16}{}%
\textbf{Poisot et al.} (2016c). Describe, understand and predict: why do
we need networks in ecology? \emph{Funct Ecol.} 30:1878--82.

\leavevmode\hypertarget{ref-PomeThom18}{}%
\textbf{Pomeranz et al.} (2018). Inferring predator-prey interactions in
food webs. \emph{Methods in Ecology and Evolution.} 0.

\leavevmode\hypertarget{ref-RoyBaxt16}{}%
\textbf{Roy et al.} (2016). Focal Plant Observations as a Standardised
Method for Pollinator Monitoring: Opportunities and Limitations for Mass
Participation Citizen Science. \emph{PLOS ONE.} 11:e0150794.

\leavevmode\hypertarget{ref-StocPois17}{}%
\textbf{Stock et al.} (2017). Linear filtering reveals false negatives
in species interaction data. \emph{Scientific Reports.} 7:45908.

\leavevmode\hypertarget{ref-StroLero14}{}%
\textbf{Strong \& Leroux}. (2014). Impact of Non-Native Terrestrial
Mammals on the Structure of the Terrestrial Mammal Food Web of
Newfoundland, Canada. \emph{PLOS ONE.} 9:e106264.

\leavevmode\hypertarget{ref-ThomGonz17}{}%
\textbf{Thompson \& Gonzalez}. (2017). Dispersal governs the
reorganization of ecological networks under environmental change.
\emph{Nature Ecology \& Evolution.} 1:0162.

\leavevmode\hypertarget{ref-TrojOles16}{}%
\textbf{Trøjelsgaard \& Olesen}. (2016). Ecological networks in motion:
micro- and macroscopic variability across scales. \emph{Funct Ecol.}
30:1926--35.

\leavevmode\hypertarget{ref-TyliMorr17}{}%
\textbf{Tylianakis \& Morris}. (2017). Ecological Networks Across
Environmental Gradients. \emph{Annual Review of Ecology, Evolution, and
Systematics.} 48:25--48.

\leavevmode\hypertarget{ref-Whit62}{}%
\textbf{Whittaker}. (1962). Classification of Natural Communities.
\emph{Botanical Review.} 28:1--239.

\leavevmode\hypertarget{ref-YeakPire14}{}%
\textbf{Yeakel et al.} (2014). Collapse of an ecological network in
Ancient Egypt. \emph{PNAS.} 111:14472--7. 
