\hypertarget{introduction}{%
\section{Introduction}\label{introduction}}

Ecological networks are a useful way to think about ecological systems
in which species or organisms interact (Heleno et al. 2014; Poisot et
al. 2016c; Delmas et al. 2018), and recently there has been an explosion
of interest in their dynamics across large temporal scales (Tylianakis
\& Morris 2017; Baiser et al. 2019), and especially along environmental
gradients (Trøjelsgaard \& Olesen 2016; Pellissier et al. 2017). As
ecosystems and climates are changing rapidly, networks are at risk of
unravelling: for example by invasion of destabilizing exotic species
them (Strong \& Leroux 2014; Magrach et al. 2017), or by a ``rewiring''
of interactions among existing species (Guiden et al. 2019; Hui \&
Richardson 2019). Simulation studies suggest that knowing the shape of
the extant network is not sufficient (Thompson \& Gonzalez 2017), and
that it needs to be supplemented by additional data on the species in
the food web, climate, and climate projection.

This renewed interest in ecological networks has prompted several
methodological efforts. First, an expansion of the analytical tools to
study ecological networks and their variation in space (Poisot et al.
2012, 2015, 2017). Second, an improvement in large-scale
data-collection, through increased adoption of molecular biology tools
(Evans et al. 2016; Eitzinger et al. 2019) and crowd-sourcing of data
collection (Pocock et al. 2015; Bahlai \& Landis 2016; Roy et al. 2016).
Finally, a surge in the development of tools that allow us to
\emph{infer} species interactions (Morales-Castilla et al. 2015) based
on limited but complementary data on existing network properties (Stock
et al. 2017), species traits (Gravel et al. 2013; Bartomeus et al. 2016;
Brousseau et al. 2017; Desjardins-Proulx et al. 2017), and environmental
conditions (Gravel et al. 2018). These latter approaches tend to perform
well in data-poor environments (Beauchesne et al. 2016), and can be
combined through ensemble modeling or model averaging to generate more
robust predictions (Pomeranz et al. 2018). The task of inferring
interactions is particularly important because ecological networks are
difficult to adequately sample in nature (Banašek-Richter et al. 2004;
Gibson et al. 2011; Chacoff et al. 2012; Jordano 2016). The common goal
to these efforts is to facilitate the prediction of network structure,
particularly over space (Poisot et al. 2016b; Gravel et al. 2018; Albouy
et al. 2019) and into the future (Albouy et al. 2014), and to appraise
the response of that structure to possible environmental changes.

These disparate methodological efforts share another important trait:
their continued success depends on state-of-the art data management.
Novel quantitative tools demand a higher volume of network data; novel
collection techniques demand powerful data repositories; novel inference
tools demand easier integration between different types of data,
including but not limited to: interactions, species traits, taxonomy,
occurrences, and local bioclimatic conditions. In short, advancing the
science of ecological networks requires us not only to increase the
volume of available data, but to pair these data with ecologically
relevant metadata. Such data should also be made available in a way that
facilitates programmatic interaction so that they can be used by
reproducible data analysis pipelines.

Poisot et al. (2016a) introduced \texttt{mangal.io} as a first step in
this direction. In the years since the tool was originally published, we
continued development of the data representation, amount and richness of
metadata, and digitized and standardized as much ecological data as we
could find. The second major release of this database contains over 1300
networks, 120000 interactions across close to 7000 taxa, and represents
what is to our best knowledge the most complete collection of species
interactions available.

Here we ask if the current mangal database is fit for the purpose of
global-scale synthesis research into ecological networks. We conclude
that interactions over most of the planet's surface are poorly
described, despite an increasing amount of available data, due to
temporal and spatial biases in data collection. In particular, Africa,
South America, and most of Asia have very sparse coverage. This suggests
that synthesis efforts on the worldwide structure or properties of
ecological networks will be weaker within these areas. To improve this
situation, we should digitize available network information and
prioritize sampling towards data-poor locations.

\hypertarget{global-trends-in-ecological-networks-description}{%
\section{Global trends in ecological networks
description}\label{global-trends-in-ecological-networks-description}}

\hypertarget{network-coverage-is-accelerating-but-spatially-biased}{%
\subsection{Network coverage is accelerating but spatially
biased}\label{network-coverage-is-accelerating-but-spatially-biased}}

\begin{figure}
\hypertarget{fig:temporal}{%
\centering
\includegraphics{figures/figure_01_a.png}
\caption{Cumulative number of ecological networks available in
\texttt{mangal.io} as a function of the date of collection. About 1000
unique networks have been collected between 1987 and 2017, a rate of
just over 30 networks a year. This temporal increase proceeds at
different rates for diferent types of networks; while the description of
food webs is more or less constant, the global acceleration in the
dataset is due to increased interest in host-parasite interactions
starting in the late 1970s, while mutualistic networks mostly started
being recorded in the early 2000s.}\label{fig:temporal}
}
\end{figure}

The earliest recorded ecological networks date back to the late
nineteenth century, with a strong increase in the rate of collection
around the 1980s (\cref{fig:temporal}). Although the volume of available
networks has increased over time, the sampling of these networks in
space has been uneven. In \cref{fig:spatial}, we show that globally,
network collection is biased towards the Northern hemisphere, and than
different types of interactions have been sampled in different places.
As such, it is very difficult to find a spatial area of sufficiently
large size in which we have networks of predation, parasitism, and
mutualism. The inter-tropical zone is particularly data-poor, either
because data producers from the global South correctly perceive massive
re-use of their data by Western world scientists as a form of scientific
neo-colonialism (as advanced by Mauthner \& Parry 2013), thereby
providing a powerful incentive \emph{against} their publication, or
because ecological networks are subject to the same data deficit that is
affecting all fields on ecology in the tropics (Collen et al. 2008). As
Bruna (2010) identified almost ten years ago, improved data deposition
requires an infrastructure to ensure they can be repurposed for future
research, which we argue is provided by \texttt{mangal.io} for
ecological interactions.

\begin{figure}
\hypertarget{fig:spatial}{%
\centering
\includegraphics{figures/figure_01_c.png}
\caption{Each point on the map corresponds to a network with parasitic,
mutualistic, and predatory interactions. It is noteworty that the
spatial coverage of these types of interactions is uneven; the Americas
have almost no parasitic network, for example. Some places have barely
been studied at all, including Africa and Eastern Asia. This
concentration of networks around rich countries speaks to an inadequate
coverage of the diversity of landscapes on Earth.}\label{fig:spatial}
}
\end{figure}

\hypertarget{different-interaction-types-have-been-studied-in-different-biomes}{%
\subsection{Different interaction types have been studied in different
biomes}\label{different-interaction-types-have-been-studied-in-different-biomes}}

Whittaker (1962) suggested that natural communities can be partitioned
across biomes, largely defined as a function of their relative
precipitation and temperature. For all networks for which the latitude
and longitude was known, we extracted the value temperature (BioClim1,
yearly average) and precipitation (BioClim12, total annual) from the
WorldClim 2 data (Fick \& Hijmans 2017). Using these we can plot every
network on the map of biomes drawn by Whittaker (1962) (note that
because the frontiers between biomes are not based on any empirical or
systematic process, they have been omitted from this analysis). In
\cref{fig:biomes}, we show that even though networks capture the overall
diversity of precipitation and temperature, types of networks have been
studied in sub-spaces only. Specifically, parasitism networks have been
studied in colder and drier climates; mutualism networks in wetter
climates; predation networks display less of a bias.

\begin{figure}
\hypertarget{fig:biomes}{%
\centering
\includegraphics{figures/figure_02.png}
\caption{List of networks across in the space of biomes as originally
presented by Whittaker (1962). Predation networks, \emph{i.e.} food
webs, seem to have the most global coverage; parasitism networks are
restricted to low temperature and low precipitation biomes, congruent
with the majority of them being in Western Europe.}\label{fig:biomes}
}
\end{figure}

Scaling up this analysis to the 19 BioClim variables in Fick \& Hijmans
(2017), we extracted the position of every network in the bioclimatic
space, conducted a principal component analysis on the scaled
bioclimatic variables, and measured their distance to the centre of this
space (\(\mathbf{0}\)). This is a measurement of the ``rarity'' of the
bioclimatic conditions in which any networks were sampled, with larger
values indicating more unique combinations (the distance was ranged to
\(]0;1]\) for the sake of interpretation). As shown in \cref{fig:ecc},
mutualistic interactions tend to have values that are higher than both
parasitism and predation, suggesting that they have been sampled in more
unique environments.

\begin{figure}
\hypertarget{fig:ecc}{%
\centering
\includegraphics{figures/figure_05_b.png}
\caption{tk}\label{fig:ecc}
}
\end{figure}

\hypertarget{some-locations-on-earth-have-no-climate-analogue}{%
\subsection{Some locations on Earth have no climate
analogue}\label{some-locations-on-earth-have-no-climate-analogue}}

Climate analogue

\includegraphics{figures/envirodistance_mutualism.png}
\includegraphics{figures/envirodistance_parasitism.png}
\includegraphics{figures/envirodistance_predation.png}

\hypertarget{conclusions}{%
\section{Conclusions}\label{conclusions}}

\hypertarget{can-we-predict-the-future-of-ecological-networks-under-climate-change}{%
\subsection{Can we predict the future of ecological networks under
climate
change?}\label{can-we-predict-the-future-of-ecological-networks-under-climate-change}}

Perhaps unsurprisingly, most of our knowledge on ecological networks is
derived from data that were collected after the 1990s
(\cref{fig:temporal}). This means that we have worryingly little
information on ecological networks before the acceleration of the
climate crisis, and therefore lack a robust baseline. Dalsgaard et al.
(2013) provide strong evidence that the extant shape of ecological
networks emerged in part in response to historical trends in climate
change. The lack of reference data before the acceleration of the
effects of climate change is of particular concern, as we may be
deriving intuitions on ecological network structure and assembly rules
from networks that are in the midst of important ecological
disturbances. Although there is some research on the response of
co-occurrence and indirect interactions to climate change (Araújo et al.
2011; Losapio \& Schöb 2017), these are a far cry from actual direct
interactions; similarly, the data on ``paleo-foodwebs'', \emph{i.e.}
from deep evolutionary time (Nenzén et al. 2014; Yeakel et al. 2014;
Muscente et al. 2018) represent the effect of more progressive change,
and may not adequately inform us about the future of ecological networks
under severe climate change. However, though we lack baselines against
which to measure the present, as a community we are in a position to
provide one for the future. Climate change will continue to have
important impacts on species distributions and interactions for at least
the next century {[}tk{]}. The Mangal database provides a structure to
organize and share network data, creating a baseline for future attempts
to monitor and adapt to biodiversity change.

Possibly more concerning is the fact that the spatial distribution of
sampled networks shows a clear bias towards the Western world,
specifically Western Europe and the Atlantic coasts of the USA and
Canada (\cref{fig:spatial}). This problem can be somewhat circumvented
by working on networks sampled in places that are close analogues of
those without direct information (almost all of Africa, most of South
America, a large part of Asia). However, \ref{fig:analog} suggests that
this approach will rapidly be limited: the diversity of bioclimatic
combinations on Earth leaves us with some areas lacking suitable
analogues. These regions are expected to bear the worst of the
socio-economical (\emph{e.g.} Indonesia) or ecological (\emph{e.g.}
polar regions) consequences of climate change. All things considered,
our current knowledge about the structure of ecological networks at the
global scale leaves us under-prepared to predict their response to a
warming world. From the limited available evidence, we can assume that
ecoservices supported by species interactions will be disrupted
(Giannini et al. 2017), in part because the mismatch between interacting
species will increase (Damien \& Tougeron 2019) alongside the climatic
debt accumulated within interactions (Devictor et al. 2012).

\hypertarget{for-what-purpose-are-global-ecological-network-data-fit}{%
\subsection{For what purpose are global ecological network data
fit?}\label{for-what-purpose-are-global-ecological-network-data-fit}}

What can we achieve with our current knowledge of ecological networks?
The overview presented here shows a large and detailed dataset, compiled
from almost every major biome on earth. It also displays our failure as
a community to include some of the most threatened and valuable habitats
in our work. Gaps in any dataset create uncertainty when making
predictions or suggesting causal relationships. This uncertainty must be
measured by users of these data, especially when predicting over the
``gaps'' in space or climate that we have identified. In this paper we
are not making any explicit recommendations for synthesis workflows.
Rather we this needs to be a collective process, a collaboration between
data collectors (who understand the deficiencies of these data) and data
analysts (who understand the needs and assumptions of network methods).

Mora et al. (2018)

Dalla Riva \& Stouffer (2015)

Dallas \& Poisot (2017)

\hypertarget{active-development-and-data-contribution}{%
\subsection{Active development and data
contribution}\label{active-development-and-data-contribution}}

This is an open-source project: all data and all code supporting this
are available on the Mangal project GitHub organization. Our hope is
that the success of this project will encourage similar efforts within
other parts of the ecological community. In addition, we hope that this
project will encourage the recognition of the contribution that software
creators make to ecological research.

One possible avenue for synthesis work, including the contribution of
new data to Mangal, is the use of these published data to supplement and
extend existing ecological network data. This ``semi-private''
ecological synthesis could begin with new data collected by authors --
for example, a host-parasite network of lake fish in Africa, or a
pollination network of hummingbirds in Brazil. Authors could then extend
their analyses by including a comparison to analogous data made public
in Mangal. After publication of the research paper, the original data
could themselves be uploaded to Mangal. This enables the reproducibility
of this particular published paper. Even more powerfully, it allows us
to build a future of dynamic ecological analyses, wherein analyses are
automatically re-done as more data get added. This would allow a sort of
continuous assessment of proposed ecological relationships in network
structure. This cycle of data discovery and reuse is an example of the
Data Life Cycle (as discussed by DataOne, {[}tk{]}) and represents one
way to practice ecological synthesis.

\hypertarget{references}{%
\section*{References}\label{references}}
\addcontentsline{toc}{section}{References}

\hypertarget{refs}{}
\leavevmode\hypertarget{ref-AlboArch19}{}%
\textbf{Albouy et al.} (2019). The marine fish food web is globally
connected. \emph{Nature Ecology \& Evolution.} 3:1153--61.

\leavevmode\hypertarget{ref-AlboVele14}{}%
\textbf{Albouy et al.} (2014). From projected species distribution to
food-web structure under climate change. \emph{Global Change Biology.}
20:730--41.

\leavevmode\hypertarget{ref-ArauRoze11}{}%
\textbf{Araújo et al.} (2011). Using species co-occurrence networks to
assess the impacts of climate change. \emph{Ecography.} 34:897--908.

\leavevmode\hypertarget{ref-BahlLand16}{}%
\textbf{Bahlai \& Landis}. (2016). Predicting plant attractiveness to
pollinators with passive crowdsourcing. \emph{Royal Society Open
Science.} 3:150677.

\leavevmode\hypertarget{ref-BaisGrav19}{}%
\textbf{Baiser et al.} (2019). Ecogeographical rules and the
macroecology of food webs. \emph{Global Ecology and Biogeography.} 0.

\leavevmode\hypertarget{ref-BanaCatt04}{}%
\textbf{Banašek-Richter et al.} (2004). Sampling effects and the
robustness of quantitative and qualitative food-web descriptors. \emph{J
Theor Biol.} 226:23--32.

\leavevmode\hypertarget{ref-BartGrav16}{}%
\textbf{Bartomeus et al.} (2016). A common framework for identifying
linkage rules across different types of interactions. \emph{Funct Ecol.}
30:1894--903.

\leavevmode\hypertarget{ref-BeauDesj16}{}%
\textbf{Beauchesne et al.} (2016). Thinking Outside the Box--predicting
Biotic Interactions in Data-poor Environments. \emph{Vie et milieu-life
and enVironment.} 66:333--42.

\leavevmode\hypertarget{ref-BrouGrav17}{}%
\textbf{Brousseau et al.} (2017). Trait-matching and phylogeny as
predictors of predator-prey interactions involving ground beetles.
\emph{Functional Ecology.}

\leavevmode\hypertarget{ref-Brun10}{}%
\textbf{Bruna}. (2010). Scientific Journals can Advance Tropical Biology
and Conservation by Requiring Data Archiving. \emph{Biotropica.}
42:399--401.

\leavevmode\hypertarget{ref-ChacVazq12}{}%
\textbf{Chacoff et al.} (2012). Evaluating sampling completeness in a
desert plant-pollinator network. \emph{J Anim Ecol.} 81:190--200.

\leavevmode\hypertarget{ref-CollRam08}{}%
\textbf{Collen et al.} (2008). The Tropical Biodiversity Data Gap:
Addressing Disparity in Global Monitoring. \emph{Tropical Conservation
Science.} 1:75--88.

\leavevmode\hypertarget{ref-DallStou15}{}%
\textbf{Dalla Riva \& Stouffer}. (2015). Exploring the evolutionary
signature of food webs' backbones using functional traits. \emph{Oikos.}
125:446--56.

\leavevmode\hypertarget{ref-DallPois17}{}%
\textbf{Dallas \& Poisot}. (2017). Compositional turnover in host and
parasite communities does not change network structure.
\emph{Ecography.}:n/a--a.

\leavevmode\hypertarget{ref-DalsTroj13}{}%
\textbf{Dalsgaard et al.} (2013). Historical climate-change influences
modularity and nestedness of pollination networks. \emph{Ecography.}
36:1331--40.

\leavevmode\hypertarget{ref-DamiToug19}{}%
\textbf{Damien \& Tougeron}. (2019). Prey-predator phenological mismatch
under climate change. \emph{Current Opinion in Insect Science.}

\leavevmode\hypertarget{ref-DelmBess18}{}%
\textbf{Delmas et al.} (2018). Analysing ecological networks of species
interactions. \emph{Biological Reviews.}:112540.

\leavevmode\hypertarget{ref-DesjLaig17}{}%
\textbf{Desjardins-Proulx et al.} (2017). Ecological interactions and
the Netflix problem. \emph{PeerJ.} 5.

\leavevmode\hypertarget{ref-Devivan12}{}%
\textbf{Devictor et al.} (2012). Differences in the climatic debts of
birds and butterflies at a continental scale. \emph{Nature Climate
Change.} 2:121--4.

\leavevmode\hypertarget{ref-EitzAbre19}{}%
\textbf{Eitzinger et al.} (2019). Assessing changes in arthropod
predator--prey interactions through DNA-based gut content
analysis---variable environment, stable diet. \emph{Molecular Ecology.}
28:266--80.

\leavevmode\hypertarget{ref-EvanKits16}{}%
\textbf{Evans et al.} (2016). Merging DNA metabarcoding and ecological
network analysis to understand and build resilient terrestrial
ecosystems. \emph{Functional Ecology.}

\leavevmode\hypertarget{ref-FickHijm17}{}%
\textbf{Fick \& Hijmans}. (2017). WorldClim 2: new 1-km spatial
resolution climate surfaces for global land areas. \emph{Int J
Climatol.}:n/a--a.

\leavevmode\hypertarget{ref-GianCost17}{}%
\textbf{Giannini et al.} (2017). Projected climate change threatens
pollinators and crop production in Brazil. \emph{PLOS ONE.} 12:e0182274.

\leavevmode\hypertarget{ref-GibsKnot11}{}%
\textbf{Gibson et al.} (2011). Sampling method influences the structure
of plant--pollinator networks. \emph{Oikos.} 120:822--31.

\leavevmode\hypertarget{ref-GravBais18}{}%
\textbf{Gravel et al.} (2018). Bringing Elton and Grinnell together: a
quantitative framework to represent the biogeography of ecological
interaction networks. \emph{Ecography.} 0.

\leavevmode\hypertarget{ref-GravPois13}{}%
\textbf{Gravel et al.} (2013). Inferring food web structure from
predator-prey body size relationships. Freckleton, ed. \emph{Methods in
Ecology and Evolution.} 4:1083--90.

\leavevmode\hypertarget{ref-GuidBart19}{}%
\textbf{Guiden et al.} (2019). Predator--Prey Interactions in the
Anthropocene: Reconciling Multiple Aspects of Novelty. \emph{Trends in
Ecology \& Evolution.} 0.

\leavevmode\hypertarget{ref-HeleGarc14}{}%
\textbf{Heleno et al.} (2014). Ecological networks: delving into the
architecture of biodiversity. \emph{Biol Lett.} 10.

\leavevmode\hypertarget{ref-HuiRich19}{}%
\textbf{Hui \& Richardson}. (2019). How to Invade an Ecological Network.
\emph{Trends in Ecology \& Evolution.} 34:121--31.

\leavevmode\hypertarget{ref-Jord16}{}%
\textbf{Jordano}. (2016). Chasing Ecological Interactions. \emph{PLOS
Biol.} 14:e1002559.

\leavevmode\hypertarget{ref-LosaScho17}{}%
\textbf{Losapio \& Schöb}. (2017). Resistance of plant--plant networks
to biodiversity loss and secondary extinctions following simulated
environmental changes. \emph{Functional Ecology.} 31:1145--52.

\leavevmode\hypertarget{ref-MagrHolz17}{}%
\textbf{Magrach et al.} (2017). Plant-pollinator networks in
semi-natural grasslands are resistant to the loss of pollinators during
blooming of mass-flowering crops. \emph{Ecography.}:n/a--a.

\leavevmode\hypertarget{ref-MautParr13}{}%
\textbf{Mauthner \& Parry}. (2013). Open Access Digital Data Sharing:
Principles, Policies and Practices. \emph{Social Epistemology.}
27:47--67.

\leavevmode\hypertarget{ref-MoraGrav18}{}%
\textbf{Mora et al.} (2018). Identifying a common backbone of
interactions underlying food webs from different ecosystems.
\emph{Nature Communications.} 9:2603.

\leavevmode\hypertarget{ref-MoraMati15}{}%
\textbf{Morales-Castilla et al.} (2015). Inferring biotic interactions
from proxies. \emph{Trends in Ecology \& Evolution.}

\leavevmode\hypertarget{ref-MuscPrab18}{}%
\textbf{Muscente et al.} (2018). Quantifying ecological impacts of mass
extinctions with network analysis of fossil communities.
\emph{PNAS.}:201719976.

\leavevmode\hypertarget{ref-NenzMont14}{}%
\textbf{Nenzén et al.} (2014). The Impact of 850,000 Years of Climate
Changes on the Structure and Dynamics of Mammal Food Webs. \emph{PLOS
ONE.} 9:e106651.

\leavevmode\hypertarget{ref-PellAlbo17}{}%
\textbf{Pellissier et al.} (2017). Comparing species interaction
networks along environmental gradients. \emph{Biol Rev Camb Philos Soc.}

\leavevmode\hypertarget{ref-PocoRoy15}{}%
\textbf{Pocock et al.} (2015). The Biological Records Centre: a pioneer
of citizen science. \emph{Biol J Linn Soc.} 115:475--93.

\leavevmode\hypertarget{ref-PoisBais16}{}%
\textbf{Poisot et al.} (2016a). mangal - making ecological network
analysis simple. \emph{Ecography.} 39:384--90.

\leavevmode\hypertarget{ref-PoisCana12}{}%
\textbf{Poisot et al.} (2012). The dissimilarity of species interaction
networks. \emph{Ecol Lett.} 15:1353--61.

\leavevmode\hypertarget{ref-PoisGrav16}{}%
\textbf{Poisot et al.} (2016b). Synthetic datasets and community tools
for the rapid testing of ecological hypotheses. \emph{Ecography.}
39:402--8.

\leavevmode\hypertarget{ref-PoisGuev17}{}%
\textbf{Poisot et al.} (2017). Hosts, parasites and their interactions
respond to different climatic variables. \emph{Global Ecol
Biogeogr.}:n/a--a.

\leavevmode\hypertarget{ref-PoisStou15}{}%
\textbf{Poisot et al.} (2015). Beyond species: why ecological
interaction networks vary through space and time. \emph{Oikos.}
124:243--51.

\leavevmode\hypertarget{ref-PoisStou16}{}%
\textbf{Poisot et al.} (2016c). Describe, understand and predict: why do
we need networks in ecology? \emph{Funct Ecol.} 30:1878--82.

\leavevmode\hypertarget{ref-PomeThom18}{}%
\textbf{Pomeranz et al.} (2018). Inferring predator-prey interactions in
food webs. \emph{Methods in Ecology and Evolution.} 0.

\leavevmode\hypertarget{ref-RoyBaxt16}{}%
\textbf{Roy et al.} (2016). Focal Plant Observations as a Standardised
Method for Pollinator Monitoring: Opportunities and Limitations for Mass
Participation Citizen Science. \emph{PLOS ONE.} 11:e0150794.

\leavevmode\hypertarget{ref-StocPois17}{}%
\textbf{Stock et al.} (2017). Linear filtering reveals false negatives
in species interaction data. \emph{Scientific Reports.} 7:45908.

\leavevmode\hypertarget{ref-StroLero14}{}%
\textbf{Strong \& Leroux}. (2014). Impact of Non-Native Terrestrial
Mammals on the Structure of the Terrestrial Mammal Food Web of
Newfoundland, Canada. \emph{PLOS ONE.} 9:e106264.

\leavevmode\hypertarget{ref-ThomGonz17}{}%
\textbf{Thompson \& Gonzalez}. (2017). Dispersal governs the
reorganization of ecological networks under environmental change.
\emph{Nature Ecology \& Evolution.} 1:0162.

\leavevmode\hypertarget{ref-TrojOles16}{}%
\textbf{Trøjelsgaard \& Olesen}. (2016). Ecological networks in motion:
micro- and macroscopic variability across scales. \emph{Funct Ecol.}
30:1926--35.

\leavevmode\hypertarget{ref-TyliMorr17}{}%
\textbf{Tylianakis \& Morris}. (2017). Ecological Networks Across
Environmental Gradients. \emph{Annual Review of Ecology, Evolution, and
Systematics.} 48:25--48.

\leavevmode\hypertarget{ref-Whit62}{}%
\textbf{Whittaker}. (1962). Classification of Natural Communities.
\emph{Botanical Review.} 28:1--239.

\leavevmode\hypertarget{ref-YeakPire14}{}%
\textbf{Yeakel et al.} (2014). Collapse of an ecological network in
Ancient Egypt. \emph{PNAS.} 111:14472--7. 
